\documentclass[class=article, crop=false]{standalone}

\usepackage{style}
\usepackage{standalone}

\begin{document}

\subsection{Type System}\label{Type System}

$V$ has a strong and static type inference system that checks a program to decide whether or not it is "`well-typed"'.
If a program is considered to be well-typed, the type system guarantees that the program will be able to be properly evaluated according to the operational semantics of $V$.
As a side-effect of checking the validity of a program, the type system can also provide the actual type of any implicitly typed expression down to its basic types, be those concrete types or variable types.

\subsubsection{Polymorphism}\label{Polymorphism}

$V$ has support for parametric Damas\hyp Milner polymorphism.
This means that functions can have their types be defined with universal quantifiers, allowing their use with any type.

For instance, take the function $count$, which counts the number of elements in a list.
This function can be defined as follows:

\smallskip

\code{let count = rec count x $\Rightarrow$ if isempty x then 0 else 1 + count (tl x) in}

\code{count (3::4::nil)}

\smallskip

In this situation, $count$ can be used with a list of any type, not only Int.
To allow this, its identifier ($count$) must have a universal association in the environment, defined as so:

\smallskip
$\forall x. \; x \; list \rightarrow \mbox{Int}$
\smallskip

The universal quantifier $\forall x$ allows the type variable $x$ to be substituted for any concrete type when the function is called.
When creating a polymorphic type, the type system must identify which type variables are free in the function type and which are bound in the environment.
This process guarantees that a polymorphic type only universally quantifies those type variables that are not already bound, while still allowing all free variables to be instantiated when the function is called.

\subsubsection{Traits}

Traits are characteristics that a type can have, defining behaviors expected of that type.
Some expressions are polymorphic in a sense that they accept certain types for their operators, but not any type.

\subsubsection{Type Inference System}

The type inference system is composed of two basic parts:
\begin{itemize}
  \item Constraint Collection
  \item Constraint Unification
\end{itemize}

Constraints are equations between type expressions, which can have both constant types and variable types.
To infer the type of a program, the type system recursively collects a set of constraints for every subexpression in that program.
This is done in a static way across the expression tree from the nodes to the root, without having to evaluate any of the expressions.
To create a valid set of constraints, the system must contain an environment, built from the root to the nodes, to ensure identifiers are properly typed.

\paragraph{Environment}
Just like the operational semantics, the type system also uses an environment to store information about identifiers.
In this case, the environment maps identifiers to type associations.
These can be either simple associations or universal associations, which are used for polymorphic functions.

\paragraph{Simple Associations}
These associate an identifier with a unique type, which can be either constant or a variable type.
When the association is called, the type is returned as-is, even if it is a variable type.

\paragraph{Universal Associations}
This association, also called a type scheme, stores a type which contains at least one variable type bound by a ``for all'' quantifier ($\forall$).
When called, this association creates a new variable type for each bound variable and returns a new instance of the type scheme.
Universal associations are used exclusively for polymorphic functions.

To create this type of association, the type system must generate a list of ``free variables'' present in the type that is to be universalized.
These are the variable types that are not present in the environment when the identifier is declared.
When these free variables are found, they are universally bound.
This ensures that only those variable types that are unbound in the environment become universally bound in the resulting association.

\paragraph{Constraint Unification}

After collecting every type constraint for the program, the type inference system attempts to unify these constraints and find a solution for them.
This solution comes in the form of type substitutions, which associate variable types to other types, and type traits, which associate variable types to sets of traits.

If the constraints cannot be unified - that is, if a conflict is found -, the program is deemed not well-typed.
Because of how the collection and unification process works, little information is given about where the problem ocurred.

\paragraph{Unification Application}

After obtaining a valid solution to the set of constraints, the type inference system applies the substitution to the type of the program.
This is done recursively until no more substitutions are found, resulting in what is called the principal type.
If there are any variable types in the principal type, the traits are applied to them, restricting the set of types that the variable types can represent.

\paragraph{Pattern Matching}

When a pattern is encountered (such as a \texttt{let} expression or function declaration), it is necessary to match the type of the pattern with the value.

To do this, a ``match'' function is defined.
It takes a pattern $p$ and a type $T$, returning a list of constraints and a mapping of identifiers to associations.

The following are the rules for the ``match'' function:

\infax[]
    {match(x, T) = \{\}, \{x \rightarrow T\}}

\infax[]
    {match(x: T_1, T_2) = \{T_1 = T_2\}, \{x \rightarrow T_1\}}

\infax[]
    {match(n, T) = \{T = Int\}, \{\}}

\infax[]
    {match(n: T_1, T_2) = \{T_1 = Int, T_2 = Int\}, \{\}}

\infax[]
    {match(b, T) = \{T = Bool\}, \{\}}

\infax[]
    {match(b: T_1, T_2) = \{T_1 = Bool, T_2 = Bool\}, \{\}}

\infax[]
    {match(c, T) = \{T = Char\}, \{\}}

\infax[]
    {match(c: T_1, T_2) = \{T_1 = Char, T_2 = Char\}, \{\}}

\infax[]
    {match(_, T) = \{\}, \{\}}

\infax[]
    {match(_: T_1, T_2) = \{T_1 = T_2\}, \{\}}

\smallskip

\infrule[]
    {X_1 \; is \; new}
    {match(nil, T) = \{X_1 list = T\}, \{\}}

\infrule[]
    {X_1 \; is \; new}
    {match(nil: T_1, T_2) = \{X_1 list = T_1, T_1 = T_2\}, \{\}}

\smallskip

\infrule[]
    {X_1 \; is \; new \andalso match(p_1, X_1) = c_1, env_1 \andalso match(p_2, X_1 list) = c_2, env_2}
    {match(p_1 :: p_2, T) = \{X_1 list = T\} \cup c_1 \cup c_2, env_1 \cup env_2}

\infrule[]
    {X_1 \; is \; new \andalso match(p_1, X_1) = c_1, env_1 \andalso match(p_2, X_1 list) = c_2, env_2}
    {match(p_1 :: p_2: T_1, T_2) = \{X_1 list = T_1, T_1 = T_2\} \cup c_1 \cup c_2, env_1 \cup env_2}

\smallskip

\infrule[]
    {\forall i \in \left[1, n\right] \; \; X_i \; is \; new \wedge match(p_i, X_i) = c_i, env_i}
    {match((p_1, ... p_n), T) = \{(X_i, ... X_n) = T\} \cup \displaystyle \bigcup_{i=1}^{n} c_i, \displaystyle \bigcup_{i=1}^{n} env_i}

\infrule[]
    {\forall i \in \left[1, n\right] \; \; X_i \; is \; new \wedge match(p_i, X_i) = c_i, env_i}
    {match((p_1, ... p_n): T_1, T_2) = \{(X_i, ... X_n) = T_1, T_1 = T_2\} \cup \displaystyle \bigcup_{i=1}^{n} c_i, \displaystyle \bigcup_{i=1}^{n} env_i}

\smallskip

\infrule[]
    {\forall i \in \left[1, n\right] \; \; X_i \; is \; new \wedge match(p_i, X_i) = c_i, env_i \andalso X_0^{\{\{l_i: X_i\}\} \forall i \in \left[1, n\right]}}
    {match(\{l_1: p_1, \dots, l_n: p_n, \dots\}, T) = \{X_0 = T\} \cup \displaystyle \bigcup_{i=1}^{n} c_i, \displaystyle \bigcup_{i=1}^{n} env_i}

\infrule[]
    {\forall i \in \left[1, n\right] \; \; X_i \; is \; new \wedge match(p_i, X_i) = c_i, env_i \andalso X_0^{\{\{l_i: X_i\}\} \forall i \in \left[1, n\right]}}
    {match(\{l_1: p_1, \dots, l_n: p_n, \dots\}: T_1, T_2) = \{X_0 = T_1, T_1 = T_2\} \cup \displaystyle \bigcup_{i=1}^{n} c_i, \displaystyle \bigcup_{i=1}^{n} env_i}

\infrule[]
    {\forall i \in \left[1, n\right] \; \; X_i \; is \; new \wedge match(p_i, X_i) = c_i, env_i}
    {match(\{l_1: p_1, \dots, l_n: p_n\}, T) = \{\{l_1: X_1, \dots, l_n: X_n\} = T\} \cup \displaystyle \bigcup_{i=1}^{n} c_i, \displaystyle \bigcup_{i=1}^{n} env_i}

\infrule[]
    {\forall i \in \left[1, n\right] \; \; X_i \; is \; new \wedge match(p_i, X_i) = c_i, env_i}
    {match(\{l_1: p_1, \dots, l_n: p_n\}: T_1, T_2) = \{\{l_1: X_1, \dots, l_n: X_n\} = T_1, T_1 = T_2\} \cup \displaystyle \bigcup_{i=1}^{n} c_i, \displaystyle \bigcup_{i=1}^{n} env_i}

\paragraph{Constraint Collection Rules}
Every expression in $V$ has a rule for constraint collection, similar to how every expression has a rule for its semantic evaluation.

\infax[T-Num]
    {\Gamma \vdash n : \mbox{Int} \; | \; \{\}}

\infax[T-Bool]
    {\Gamma \vdash b : \mbox{Bool} \; | \; \{\}}

\infax[T-Char]
    {\Gamma \vdash c : \mbox{Char} \; | \; \{\}}

\infrule[T-Ident]
    {\Gamma \left(x\right) = T}
    {\Gamma \vdash x : T \; | \; \{\}}

\bigskip

\infrule[T-Tuple]
    {\forall \; k \in \left[1, n\right] \; \; \Gamma \vdash e_k : T_k \; | \; C_k}
    {\Gamma \vdash (e_1, \; \dots \; e_n) : (T_1, \; \dots \; T_n) \; | \; C_1 \cup \cdots C_n }

\bigskip

\infrule[T-Record]
    {\forall \; k \in \left[1, n\right] \; \; \Gamma \vdash e_k : T_k \; | \; C_k}
    {\Gamma \vdash \{l_1: e_1, \; \dots \; l_n: e_n\} : \{l_1: T_1, \; \dots \; l_n: T_n\} \; | \; C_1 \cup \cdots C_n }

\infrule[T-RecordAccess]
    {\Gamma \vdash e_1 : T_1 \; | \; C_1 \andalso \Gamma \vdash e_2 : T_2 \; | \; C_2 \andalso X_1^{\{\{l: T_1\}\}} \; is \; new}
    {\Gamma \vdash \#l \; e_1 \; e_2 : (T_1, T_2) \; | \; C_1 \cup C_2 \cup \{X_1 = T_2\}}

\bigskip

\infrule[T-Nil]
    {X_1 \; is\; new}
    {\Gamma \vdash nil : X_1 \; list \; | \; \{\}}

\infrule[T-List]
    {\Gamma \vdash e_1 : T_1 \; | \; C_1 \andalso \Gamma \vdash e_2 : T_2\; | \; C_2}
    {\Gamma \vdash e_1 :: e_2 : T_1 \; list \; | \; C_1 \cup C_2 \cup \{T_1 \; list = T_2\}}

\bigskip

\infrule[T-+]
    {\Gamma \vdash e_1 : T_1 \; | \; C_1 \andalso \Gamma \vdash e_2 : T_2 \; | \; C_2}
    {\Gamma \vdash e_1 + e_2 : \mbox{Int} \; | \; C_1 \cup C_2 \cup \{T_1 = \mbox{Int}; T_2 = \mbox{Int}\}}

\infrule[T--]
    {\Gamma \vdash e_1 : T_1 \; | \; C_1 \andalso \Gamma \vdash e_2 : T_2 \; | \; C_2}
    {\Gamma \vdash e_1 - e_2 : \mbox{Int} \; | \; C_1 \cup C_2 \cup \{T_1 = \mbox{Int}; T_2 = \mbox{Int}\}}

\infrule[T-$\ast$]
    {\Gamma \vdash e_1 : T_1 \; | \; C_1 \andalso \Gamma \vdash e_2 : T_2 \; | \; C_2}
    {\Gamma \vdash e_1 * e_2 : \mbox{Int} \; | \; C_1 \cup C_2 \cup \{T_1 = \mbox{Int}; T_2 = \mbox{Int}\}}

\infrule[T-$\div$]
    {\Gamma \vdash e_1 : T_1 \; | \; C_1 \andalso \Gamma \vdash e_2 : T_2 \; | \; C_2}
    {\Gamma \vdash e_1 \div e_2 : \mbox{Int} \; | \; C_1 \cup C_2 \cup \{T_1 = \mbox{Int}; T_2 = \mbox{Int}\}}

\bigskip

\infrule[T-$=$]
    {\Gamma \vdash e_1 : T_1 \; | \; C_1 \andalso \Gamma \vdash e_2 : T_2 \; | \; C_2 \andalso X_1^{\{Equatable\}} \; is \; new}
    {\Gamma \vdash e_1 = e_2 : \mbox{Bool} \; | \; C_1 \cup C_2 \cup \{T_1 = T_2; X_1^{\{Equatable\}} = T_2\}}

\infrule[T-$\neq$]
    {\Gamma \vdash e_1 : T_1 \; | \; C_1 \andalso \Gamma \vdash e_2 : T_2 \; | \; C_2 \andalso X_1^{\{Equatable\}} \; is \; new}
    {\Gamma \vdash e_1 \neq e_2 : \mbox{Bool} \; | \; C_1 \cup C_2 \cup \{T_1 = T_2; X_1^{\{Equatable\}} = T_2\}}

\bigskip

\infrule[T-$<$]
    {\Gamma \vdash e_1 : T_1 \; | \; C_1 \andalso \Gamma \vdash e_2 : T_2 \; | \; C_2 \andalso X_1^{\{Orderable\}} \; is \; new}
    {\Gamma \vdash e_1 < e_2 : \mbox{Bool} \; | \; C_1 \cup C_2 \cup \{T_1 = T_2; X_1^{\{Orderable\}} = T_2\}}

\infrule[T-$\leq$]
    {\Gamma \vdash e_1 : T_1 \; | \; C_1 \andalso \Gamma \vdash e_2 : T_2 \; | \; C_2 \andalso X_1^{\{Orderable\}} \; is \; new}
    {\Gamma \vdash e_1 \leq e_2 : \mbox{Bool} \; | \; C_1 \cup C_2 \cup \{T_1 = T_2; X_1^{\{Orderable\}} = T_2\}}

\infrule[T-$>$]
    {\Gamma \vdash e_1 : T_1 \; | \; C_1 \andalso \Gamma \vdash e_2 : T_2 \; | \; C_2 \andalso X_1^{\{Orderable\}} \; is \; new}
    {\Gamma \vdash e_1 > e_2 : \mbox{Bool} \; | \; C_1 \cup C_2 \cup \{T_1 = T_2; X_1^{\{Orderable\}} = T_2\}}

\infrule[T-$\geq$]
    {\Gamma \vdash e_1 : T_1 \; | \; C_1 \andalso \Gamma \vdash e_2 : T_2 \; | \; C_2 \andalso X_1^{\{Orderable\}} \; is \; new}
    {\Gamma \vdash e_1 \geq e_2 : \mbox{Bool} \; | \; C_1 \cup C_2 \cup \{T_1 = T_2; X_1^{\{Orderable\}} = T_2\}}

\bigskip

\infrule[T-App]
    {\Gamma \vdash e_1 : T_1 \; | \; C_1 \andalso \Gamma \vdash e_2 : T_2 \; | \; C_2  \andalso X_1 \; is \; new}
    {\Gamma \vdash e_1 \; e_2 : X \; | \; C_1 \cup C_2 \cup \{T1 = T_2 \rightarrow X_1}

\bigskip

\infrule[T-Fn]
    {X \; is \; new \andalso match(p, X) = C, env \andalso env \cup \Gamma \vdash e : T_1 \; | \; C_1}
    {\Gamma \vdash \texttt{fn} \; p \Rightarrow e : X \rightarrow T_1 \; | \; C \cup C_1}

\infrule[T-Rec]
    {X \; is \; new \andalso match(p, X) = C, env\\
    \{x \rightarrow (X \rightarrow T)\} \cup env \cup \Gamma \vdash e : T_1 \; | \; C_1}
    {\Gamma \vdash \texttt{rec} \; x:T \; \; p \Rightarrow e : X \rightarrow T_1 \; | \; C \cup C_1 \cup \{T_1 = T\}}

\infrule[T-Rec2]
    {X_1 \; is \; new \andalso X_2 \; is \; new \andalso match(p, X_1) = C, env\\
     \{x \rightarrow X_2\} \cup env \cup \Gamma \vdash e : T_1 | C_1}
    {\Gamma \vdash \texttt{rec} \; x \; p \Rightarrow e : X_1 \rightarrow T_1 \; | \; C \cup C_1 \cup \{X_2 = X_1 \rightarrow T_1\}}

\bigskip

\infrule[T-Let]
    {\Gamma \vdash e_1 : T_1 \; | \; C_1 \andalso match(p, T_1) = C, env\\
    env \cup \Gamma \vdash e_2 : T_2 \; | \; C_2}
    {\Gamma \vdash \texttt{let} \; p = e_1 \; \texttt{in} \; e_2 : T_2 \; | \; C \cup C_1 \cup C_2}

\bigskip

\infrule[T-Match]
    {\Gamma \vdash e : T \; | \; C \andalso X_1 \; is \; new\\
    \forall j \in \left[1..n\right] multiMatch(T, X_1, \Gamma, match_j) = C_j}
    {\Gamma \vdash \texttt{match} \; e \; \texttt{with} \; match_1, ... \; match_n : X_1 \; | \; C \cup \displaystyle \bigcup_{i=1}^{n} C_i}

\medskip

\infrule[]
  {match(p, T_1) = C, \Gamma_1 \andalso \Gamma_1 \cup \Gamma \vdash e : T_3 \; | \; C_3}
  {multiMatch(T_1, T_2, \Gamma, p \rightarrow e) = C \cup C_3 \cup \{T_3 = T_2\}}

\infrule[]
  {match(p, T_1) = C, \Gamma_1 \andalso \Gamma_1 \cup \Gamma \vdash e_1 : T_3 \; | \; C_3 \andalso \Gamma_1 \cup \Gamma \vdash e_2 : T_4 \; | \; C_4}
  {multiMatch(T_1, T_2, \Gamma, p \; \mbox{when} \; e_1 \rightarrow e_2) = C \cup C_3 \cup C_4 \cup \{T_3 = Bool, T_4 = T_2}


\bigskip

\infrule[T-Raise]
    {X_1 \; is\; new}
    {\Gamma \vdash raise : X_1 \; | \; \{\}}

\end{document}
